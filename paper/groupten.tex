\documentclass[modern,letterpaper]{aastex61}

% to-do list
% ----------
% - add items here
% style notes
% -----------
% - This file generates by Makefile; don't be typing ``pdflatex'' or some bullshit.
% - Line break between sentences to make the git diffs readable.
% - Use \, as a multiply operator.
% - Reserve () for function arguments; use [] or {} for outer shit.
% - Use \sectionname not Section, \figname not Figure, \documentname not Article or Paper or paper.

% packages
\definecolor{cbblue}{HTML}{3182bd}
\usepackage{microtype}  % ALWAYS!
\usepackage{amsmath,amssymb}
\usepackage{tikz}
\usepackage{aas_macros}
\hypersetup{backref,breaklinks,colorlinks,urlcolor=cbblue,linkcolor=cbblue,citecolor=black}
\graphicspath{{.}{figures/}{../notebooks/plots/}}

% define macros for text
\newcommand{\project}[1]{\textsl{#1}}
\newcommand{\acronym}[1]{{\small{#1}}}
\newcommand{\hipparcos}{Hipparcos}
\newcommand{\gaia}{\project{Gaia}}
\newcommand{\rave}{\project{\acronym{RAVE}}}
\newcommand{\apogee}{\project{\acronym{APOGEE}}}
\newcommand{\tmass}{\project{\acronym{2MASS}}}
\newcommand{\allwise}{\project{\acronym{allWISE}}}
\newcommand{\panstarrs}{\project{\acronym{PanSTARRS}}}
\newcommand{\galex}{\project{\acronym{GALEX}}}
\newcommand{\rosat}{\project{\acronym{ROSAT}}}
\newcommand{\documentname}{Article}
\newcommand{\sectionname}{Section}
\newcommand{\figname}{Figure}
\newcommand{\eqname}{Equation}
\newcommand{\dr}{\acronym{DR1}}
\newcommand{\tgas}{\acronym{TGAS}}
\newcommand{\etal}{\textit{et al}.}

\newcommand{\objname}{moving group}
\newcommand{\groupTen}{Group 10}
\newcommand{\groupSixteen}{Group 16}

% define macros for math
\newcommand{\given}{\,|\,}
\newcommand{\normal}{{\mathcal{N}}}
\newcommand{\uniform}{{\mathcal{U}}}
\newcommand{\dd}{\mathrm{d}}
\newcommand{\transp}[1]{{#1}^{\!\mathsf{T}}}
\newcommand{\inv}[1]{{#1}^{-1}}
\newcommand{\bs}[1]{\boldsymbol{#1}}
\newcommand{\vperp}{\bs{v}^\perp}
\newcommand{\propm}{\bs{\mu}}
\newcommand{\mat}[1]{\mathbf{#1}}
\renewcommand{\vec}[1]{\bs{#1}}
\newcommand{\kms}{\ensuremath{\rm km~s^{-1}}}
\newcommand{\msun}{{\rm M}_\odot}
\newcommand{\data}{\mathrm{data}}
\newcommand{\snr}{[S/N]_\varpi}
\newcommand{\eye}{\mathbb{I}}
\newcommand{\absdvtan}{\ensuremath{|\Delta\vec v_\mathrm{t}|}}
\newcommand{\estimates}{\ensuremath{\{\hat{\varpi_i},\hat{\mu_{\alpha,i}},\hat{\mu_{\delta,i}},\hat{v_{r,i}}\}}}

\newcommand{\ra}{\text{R.A.}}
\newcommand{\dec}{\text{Decl.}}
\newcommand{\pmra}{\ensuremath{\mu_\alpha^*}}
\newcommand{\pmdec}{\ensuremath{\mu_\delta}}
\newcommand{\masyr}{\ensuremath{\mathrm{mas}\,\mathrm{yr}^{-1}}}

\definecolor{crimson}{rgb}{0.79, 0.0, 0.09}
\newcommand{\todo}[1]{{\color{crimson}#1}}

\newcommand{\groupDistanceEstimate}{\ensuremath{100~\mathrm{pc}}}
\newcommand{\groupAgeEstimate}{\ensuremath{500~\mathrm{Myr}}}
\newcommand{\totalNumberOfCandidates}{\ensuremath{29}}
\newcommand{\meanVelocityOpt}{\ensuremath{(u_x,\,u_y,\,u_z) = (-3.9,\,7.0,\,-3.8)~\kms}}
\newcommand{\sigVelocityOpt}{\ensuremath{0.42~\kms}}
\newcommand{\nstarsInRegion}{3413}
\newcommand{\bprp}{\ensuremath{\mathrm{BP}-\mathrm{RP}}}

\begin{document}\sloppy\sloppypar\raggedbottom\frenchspacing % trust me

\title{
  Rediscovery of a nearby young, coeval moving group: a cluster in disruption?
}

\author[0000-0001-7790-5308]{Semyeong Oh \etal}
\affiliation{Department of Astrophysical Sciences,
             Princeton University, Princeton, NJ 08544, USA}

\correspondingauthor{Semyeong Oh}
\email{semyeong@astro.princeton.edu}

\begin{abstract}

  We confirm and characterize a nearby ($d \approx \groupDistanceEstimate$),
  young moving group discovered with \gaia\ DR 1 astrometry using the updated
  and extended data from \gaia\ DR 2 in combination with other data from the
  literature.
  This group, which notably includes 81 Uma and 84 Uma, spans a large area
  on sky ($(\Delta\mathrm{R.A.},\,\Delta\mathrm{Decl.})\approx(39,\,18)$~deg),
  thus giving a well-constrained astrometric radial velocity from geometry alone.
  We model the proper motions and radial velocities of the candidate members
  with a simple isotropic Gaussian velocity distribution, and find that the
  velocity dispersion is very small (\todo{X}) verifying that they are comoving.
  The morphology of the group in the Galactic $X-Y$ coordinates shows an
  interesting gap while highly contentrated in $Z$ direction.
  We discuss possible origins of the gap in relation to the disruption of the group.

\end{abstract}

\section{Introduction} % (fold)
\label{sec:introduction}

Comoving, coeval stars in the Solar neighborhood are valuable for
many different fields of astrophysics.
They are important benchmarks for our understanding of stellar physics
and for calibrating stellar measurements among different mass ranges.
In the general picture that stars form in groups and remain
kinematically coherent until they are dissolved into the field population,
a detailed census of these objects is closely related to the recent star
formation history in the local volume as well as the dynamics of their disruption.

Each star cluster or coeval moving group are also interesting dynamical entities
themseleves as their dynamical evolution is intimately tied to their internal
structure and external perturbation such as giant molecular clouds.

We are entreing a new era in charting the kinematic substructure and stellar
populations in the Solar neighborhood with the recent second data release (DR~2)
of \gaia\ \citep{2018arXiv180409365G}, which provides parallaxes, proper motions
and precise two-band photometry of stars brighter than $G=20$~mag as well as
radial velocities for stars brighter than 12~mag.
Previously, we performed a search for comoving pairs using the Tycho-Gaia
Astrometric Solution (\tgas), a subset of \gaia\ DR~1 which provides astrometric
measurements for 2~million source in Tycho-2 catalog ($G<12$~mag).
By looking for comoving pairs blindly, we have naturally recovered
many of the other larger known comoving stellar groups, ranging from
loose associations to open clusters.
However, we have also found some new large and small groups that
were not previously idendified.
% TODO: discuss Faherty 2018

Here, we confirm one of the new coeval moving groups
discovered in \citet{2017AJ....153..257O} as Group 10
with \gaia\ DR 2 and present an extended analysis of the properties.
This group stands out as the largest nearby ($\approx 100$~pc) group with little
prior discussion in the literature.
We present the data and candidate member selection in
\sectionname~\ref{sec:data}, discuss the ages, kinematic modelling and
morphology of the group in \sectionname~\ref{sec:analysis},
and summarize in \ref{sec:discussion}.


\subsection{Previous identifications}
\label{subsec:history}

Moving group identifications are often scattered throughout the literature. We
reviewed a (non-exhaustive) bibliography associated with all \tgas\ sources in
\groupTen\ from our previous selection \citep{2017AJ....153..257O} using the
Simbad database, and note previous identifications of this group.

Three A-type stars that were selected in \citet{2017AJ....153..257O} as potential
members of \groupTen,
HIP 66198 (81 Uma), HIP 67231 (84 Uma), and HIP 67005, are part of an open cluster
suggested by \citet{1977ATsir.969....7L} and dubbed Latyshev 2 by Archinal \& Hynes.
The reality of this putative cluster, composed of 7 A-type stars
(HIP 66198,
HIP 67231,
HIP 67005,
HIP 67848,
HIP 66738,
TYC 3851-606-1,
TYC 3469-497-1)
was dismissed in a recent review by \citet{2016IAUS..314...21M}.
However, here we report that there indeed is a comoving group containing
some of the stars in Latyshev 2, and demonstrate
its physical reality more convincingly.
%
We found that two of the \hipparcos\ stars in Latyshev 2,
HIP 67848 (86 Uma) and HIP 66738 (83 Uma),
were missing in the \tgas\ catalog which was the parent sample of our previous search while
their proper motions have been determined by \hipparcos.
Motivated by this, we looked for any bright members of the group that we may
have missed using \hipparcos\ stars that are not present in \tgas\ (\todo{see
section X}).

\section{Data \& Sample}
\label{sec:data}

\begin{figure}
  \includegraphics[width=0.95\linewidth]{g10_sky_pm.pdf}
  \caption{Distribution of stars in the local neighborhood of \groupTen\ on sky and in proper-motion space.}
  \label{fig:distributions}
\end{figure}

We use the astrometric, photometric, and spectroscopic parameters of
\gaia\ DR~2.
Based on the angular and parallactic spread of this group from the previous
discovery using \tgas, we query the same region in \gaia\ DR 2 with generous
cuts applied in \ra, \dec, and parallax.
We go out to 20 degrees and 9 degrees from the mean position in \ra\ and \dec\
respectively, and a nominal 20~pc in distance around the median parallax.
There are \nstarsInRegion\ \gaia\ sources within these cuts.
Figure~\ref{fig:distributions} shows the distribution of these sources on sky
and in proper-motion space.
While there is no discernable concentration of sources on sky
(\figname~\ref{fig:distributions}a),
a clear overdensity is found in proper-motion space at
$(\pmra,\,\pmdec)\approx(-16.8,\,-3)~\masyr$ in agreement with the previous
identification using \tgas.
Of course, the sources in this overdensity do not have exactly the same proper
motion due to geometric projection.
The span of this overdensity in \figname~\ref{fig:distributions}b
mainly corresponds to the spread in parallaxes,
with sources with larger parallaxes having larger proper motions.

In order to select candidate members of the group,
we fit a Gaussian mixture model to the zoom-in box indicated in
\figname~\ref{fig:distributions}b in $(v_\alpha^*,\,v_\delta)$ space
with two components, one for the overdensity corresponding to the group and
one broad component for the background.
We select 194 sources with higher probability of belonging to the overdensity
component as candidate members of the group.
These sources, along with $1\sigma$ ellipse of each component of the mixture model,
are indicated in \figname~\ref{fig:distributions}c.
We expect this selection to be neither complete nor without contamination.
However, given the clear overdensity in proper-motion space within this parallax slice
of the group, we expect the contamination to be low.
We think this selection is sufficient for a first attempt to examine the general
properties of the group.
Of 194 sources selected, there are 31 sources that have radial velocities (RVs)
measured.
The median and standard deviation of the velocities of 31 sources with RVs
in the Galactic coordinates
are $(U,\,V,\,W) = (-3.6,\,-9.1,\,-1.6)~\kms$ and
$(\sigma_U,\,\sigma_V,\,\sigma_W) = (0.60,\,0.88,\,1.23)~\kms$,
aptly described as `comoving'.
We note that 95\% of the selected candidate members have parallax
signal-to-noise ratios above 24 with median signal-to-noise ratio of 170.


\section{Analysis}
\label{sec:analysis}

\subsection{Color-magnitude diagrams and isochrone ages}

\begin{figure}
  \includegraphics[width=0.95\linewidth]{bp_rp_G.pdf}
  \caption{
    Color-magnitude diagrams of candidate \groupTen\ members.
    Panel (a) and (c-f) show the color-magnitude diagram using \gaia\ (BP, RP and $G$),
    \tmass\ ($J$), \allwise\ ($W1$), and \panstarrs\ ($r_\mathrm{PS}$ and $i_\mathrm{PS}$)
    magnitudes. Panel (b) shows the astrometric excess noise from \gaia's astrometric
    solution vs. BP-RP color. The sources for which the excess noise is a significant
    outlier to the general trend are colored red.
    Some of the sources with bad BP, RP photometry are marked by `x'.
    Many of these seem to be low-mass stars further down the main sequence (d-f).
  }
  \label{fig:cmd}
\end{figure}

\begin{figure}
  \centering
  \includegraphics{age_DavidHillenbrand2015.pdf}
  \caption{Age estimates for five candidate members by \citet{2015ApJ...804..146D}.
    For each star, the black vertical line is the mode of the posterior
    distribution, and the gray band and line marks 68\% and 95\% confidence
    interval.
  }
  \label{fig:age_DavidHillenbrand2015}
\end{figure}

Figure~\ref{fig:cmd}a shows the color-magnitude diagrams
of the candidate group members using the \gaia\ photometry.
The absolute $G$ magnitude $M_G$ is calculated as $G + 5\log(\varpi) -10$ where
$\varpi$ is parallax in mas.
The group is at a high galactic latitude (median $b=55$~degrees),
and the extinction is very low.
We checked the dust maps of \citet{1998ApJ...500..525S} and
\citet{2017ApJ...846...38L}
at the location of candidate members and found the median reddening to be
$E(B-V)\lesssim 0.012$~mag although there are some variations across the region with
the maximum reddening of 0.025~mag.
We applied no correction for dust to the data or isochrone models.

Compared to the distribution of stars in the entire search region (gray circles),
the candidate members (black circles) resemble a clean single-age population.
In order to estimate an approximate age of the group,
we compared the distribution of candidate members in the color-magnitude diagram
to MIST \citep[][with rotation]{2016ApJ...823..102C} and
PARSEC isochrones \citep[v1.2S;][]{2012MNRAS.427..127B,2015MNRAS.452.1068C}
of solar metallicity visually,
and found $\log(\mathrm{age}) \approx 8.4$ (250~Myr) to be a good fit.
A range of $\log(\mathrm{age}) = (8.2-8.55)$ should encapsulate the plausible
age of the group judging from the disagreement in lower-main sequence for
younger ages (with PARSEC models) and the main-sequence turn-off for older ages.
Isochrones of $\log(\mathrm{age}) \approx 8.4$
from MIST (black) and PARSEC (blue) models are plotted in \figname~\ref{fig:cmd}a
for comparison.
The discrepancy between the MIST models and the data for low-mass
($M\lesssim0.6~m_\odot$) stars is a well-known problem seen in many young open
clusters or associations and generic in many stellar models, and is attributed
to incomplete atomic and molecular line opacity data.
This is remedied in PARSEC models by revising and calibrating the boundary
conditions to match the observed star clusters
\citep{2014MNRAS.444.2525C}, leading to a better agreement with the data.
A more careful modelling is required to properly infer the age and metallicity
of the group.

At the very faint end, the \gaia\ BP-RP colors are may be incorrect due to
contamination from nearby sources as no deblending was performed for BP and RP
bands in DR~2 \citep{2018arXiv180409368E}.
Following \citet{2018arXiv180409378G}, we used emperical cuts to the
\texttt{phot\_bp\_rp\_excess\_factor} as a function of BP-RP color in order to
flag these sources which are marked as `\texttt{x}' in all panels of
Figure~\ref{fig:cmd}.
In order to illuminate the nature of these sources, we use four other deeper
survey magnitudes (\tmass\ $J$, \allwise\ $W1$ and \panstarrs\ $r$ and $y$) with
\gaia\ $G$ band magnitudes.
%
Many of these sources seem to be lower-mass stars further down the main-sequence
with even redder colors.

For five \hipparcos\ stars of the candidate members,
an independent age determination by \citet{2015ApJ...804..146D} exists.
The stars were treated as field stars and modelled individually.
We used the cross-match to Tycho-2 provided in the Gaia Archive
in order to retrieve the \hipparcos\ identifiers.
Figure~\ref{fig:age_DavidHillenbrand2015} shows the posterior distributions
of their age estimates.
Except for one star, HIP 69275, with a notable disagreement with the rest,
four out of five stars have a consistent most-probable age of $202-214$~Myr
(mode of the posterior distribution) and overlapping posterior distributions.
This is in good agreement with the visually-determined isochrone age.

\subsection{Chromospheric activity}

\begin{figure}
  \centering
  \includegraphics{nuv_xray.pdf}
  \caption{$\mathrm{NUV}-\mathrm{G}$ colors of 62 \galex\ cross-matched
    candidate members in \groupTen\ as a function of their \bprp\ color.
    }
  \label{fig:nuv}
\end{figure}



As an independent sign of their youth, we checked the UV and X-ray detection of
the candidate members by cross-mathing to the \galex\ catalog
\citep{2005ApJ...619L...1M} and the \rosat\ all-sky survey
\citep[2RXS;][]{2016A&A...588A.103B}.
We used $2\arcsec$ and $20\arcsec$ search radius for \galex\ and 2RXS respectively.
When there are multiple \galex\ matches to a source, we choose the one with
the smallest NUV magnitude error.
We find that 62 of our candidate members ($\sim30$\%) have corresponding \galex\
UV detection, of which 10 are also X-ray sources.
There are additional 4 X-ray detected candidate members.
Their $\mathrm{NUV}-\mathrm{G}$ colors (\ref{fig:nuv}) show UV excess commonly
seen in young stars upto ages of a few Myr old, consistent with the isochrone age.


\subsection{Mean velocity and velocity dispersion}
\label{sec:fitting}

In this section, we derive the mean velocity and velocity dispersion of the
group using either only parallaxes and proper motions or full 6D information
including radial velocities.
Although the original selection was done by selecting comoving pairs, we can
also model the proper motions of the stars in the group as drawn from a single
mean velocity and a dispersion.
If the stars are indeed comoving, the dispersion should come out small.

\todo{rephrase:
This is essentially the moving cluster method or the convergent point method,
but instead of taking procedural steps to confirm or find the existence of a
convergent point, we take a forward modelling approach with a probabilistic basis.}

\todo{CITE: Lindegren etc.}



The components of the model are:
\begin{itemize}
  \item We assume that the velocity $\vec{v}_i$ of the star $i$ is drawn from
    a single Gaussian component with mean (group) velocity $\vec{u}$ and
    an isotropic dispersion $\sigma_{u}$:
    $$\vec{v}_i \sim \normal(\vec{u},\,\sigma_{u}).$$
    Then, the proper motion of star $i$ is
    $\transp{(\pmra,\,\pmdec)} = \mat{M}_i(\alpha_i,\,\delta_i) \vec{v}_i / r_i$ where
    $\mat{M}_i$ is the rotation matrix that transforms the equatorial
    rectangular coordinates to the tangential coordinates at the location
    $(\mathrm{R.A.},\,\mathrm{Decl.}) = (\alpha_i,\,\delta_i)$.

  \item We assume that the noise model for the \gaia\ data is Gaussian, and
    the covariance matrix $\mat{C_i}$ is given and fixed:
    $$(\tilde\varpi_i,\,\tilde\mu_{\alpha,i}^*,\,\tilde\mu_{\delta,i}) \sim
      \normal((\varpi_i,\,\mu_{\alpha,i}^*,\,\mu_{\delta,i}),\,\mat{C_i})$$

  \item \emph{Priors}:
    We assume an uninformative uniform prior in distance and a broad Gaussian
    prior $\vec{u} \sim \normal(0,\,100~\kms)$ and $\sigma_u \sim \normal(0, 50~\kms)$
    for the mean velocity $\vec{u}$ and velocity dispersion $\sigma_u$.
    Given that the parallax signal-to-noise is very high (median of 170),
    the distance prior has little effect on the fit.
\end{itemize}

This can be naturally extended to include radial velocities of each star.
In this case, the radial velocity is just $v_{r,i} =
\hat{\vec{u}}_r(\alpha_i,\,\delta_i) \cdot \vec{v}_i$
where $\hat{\vec{u}}_r(\alpha_i,\,\delta_i)$ is the unit radial vector
and the covariance of the noise model
is extended as $\mathrm{diag}(\mat{C_i},\,\sigma_{v_r}^2)$.
We refer the model using proper-motions only of all candidate member stars as `PM only'
and the model using a subset of 38 stars with RVs as `PM+RV' respectively.
For each case, we first do an optimization using L-BFGS algorithm starting from
the initial values $d_{i,0} = 1/{\tilde \varpi_i}$, a random mean velocity
$\vec{u}_0$ and $\sigma_u=10$~\kms.
We then sample the posterior distribution of $d_i$, $\vec{u}$ and $\sigma_u$
starting from this optimum.


\begin{figure}
  \includegraphics[width=0.95\linewidth]{isotropic.pdf}
  \caption{Fit result}
  \label{fig:fit}
\end{figure}

\figname~\ref{fig:fit} summarizes and compares the fit result for the two cases.
Because this group spans a large area on sky, all three components of the mean
velocity is well-contrained geometrically using proper-motions only
(black density and contours).
The mean and standard deviation of the posterior samples of the mean velocity $\vec{u}$
is $(u_x,\,u_y,\,u_z) = (-3.82\pm0.19,\,6.84\pm0.14,\,-4.00\pm0.34)$ in
equatorial coordinates.
The velocity dispersion is very small with the mean of the posterior samples at
$\approx 0.73$~\kms.
The fit result of `PM only' is in good agreement with `PM+RV' case
with the mean velocity of $(-3.02\pm0.13,\,7.41\pm0.14,\,-5.89\pm0.13)$ and
the mean dispersion of $0.82$~\kms (orange density and contours).
While the differences between the two models is significant formally,
the difference is very small $<1$~\kms.

\subsection{Morphology}

\begin{figure}
  \includegraphics[width=0.95\linewidth]{orbit_morphology.pdf}
  \caption{Distribution of candidate members in \groupTen\ in
    the Galactic coordinates.}
  \label{fig:orbit_morphology}
\end{figure}

The distribution of candidate members show no discernable `center' or
contentration in the Galactic $X$-$Y$ coordinates
(Figure~\ref{fig:orbit_morphology}) while it seems highly concentrated in $Z$
direction around the median height of $\approx 82$~pc above the Galactic plane.
The nominal standard deviation is $6.7$, $10$, and $4.3$ pc in $X$, $Y$, and $Z$
coordinates respectively.
Furthermore, the candidate members seem to be divided in their $X$-$Y$
distribution, leaving a band of gap that is tilted with respect to the direction
of Galactic rotation.
The bottom panels of \figname~\ref{fig:orbit_morphology} shows the
nearly-circular orbit of the group integrated backwards for $300$~Myr in a Milky
Way potential \citep{2015ApJS..216...29B} using the mean velocity derived from
`PM+RV' model fitting in \sectionname{sec:fitting} and the median position
marked by black `x' in  the top panels of the same \figname.

We can roughly estimate the total mass of the group using BP-RP colors
and a model isochrone assuming an initial mass function (IMF).
We interpolate the BP-RP color-initial mass relation of a PARSEC isochrone of
solar metallicity and $\log\mathrm{age}=8.4$.
We select stars in a color range $0<\mathrm{BP-RP}<3$ which corresponds to
initial masses of $2.1~M_\odot$ and $0.24~M_\odot$.
For the Kroupa initial mass function (CITE),
the total mass is 2.4 times the mass within this color (mass) range.
Since the total mass of 142 candidate members within this color range is
$82~M_\odot$, the total mass is estimated to be $\approx200~M_\odot$.

We speculate on the origin of the morphology of the group in relation to
dynamical disruption of a star cluster.
Star clusters that survive the infant mortality are thought to be eventually
disrupted by the Galactic tidal field or encounters with Giant Molecular Clouds
(GMCs).
As stars in a cluster gain energy from the external perturbation,
they escape primarily through $L_1$ and $L_2$ Lagrange points (Heggie 2001),
which are separated by $2r_J$ where $r_J$ is the tidal (or Jacobi) radius radius.
For a cluster like \groupTen\ in the Solar neighborhoood with nearly circular orbit,
the tidal radius due to the Galactic tides is
\begin{equation}
  \begin{split}
    r_J &= \left(\frac{G M}{2 (V_G/R_G)^2}\right)^{1/3}\\
        &= 8.5~\mathrm{pc} \left(\frac{M}{200~M_\odot}\right)^{1/3}
          \left(\frac{8.3~\mathrm{kpc}}{R}\right)^{2/3}
          \left(\frac{V}{220~\kms}\right)^{2/3}
  \end{split}
\end{equation}
This is roughly consistent with the size of the gap, $\lesssim 20$~pc.


If the group was disrupted by a GMC,



\subsection{Comparison to other groups}



\section{Summary \& Discussion}
\label{sec:discussion}

We confirmed a candidate nearby comoving group first detected


\acknowledgements % simbad
This research has made use of the SIMBAD database,
operated at CDS, Strasbourg, France.
% ads
This research has made use of NASA's Astrophysics Data System.
% gaia
This work has made use of data from the European Space Agency (ESA) mission
{\it Gaia} (\url{https://www.cosmos.esa.int/gaia}), processed by the {\it Gaia}
Data Processing and Analysis Consortium (DPAC,
\url{https://www.cosmos.esa.int/web/gaia/dpac/consortium}). Funding for the DPAC
has been provided by national institutions, in particular the institutions
participating in the {\it Gaia} Multilateral Agreement.
% gaia sprints


\bibliography{refs}

\end{document}
