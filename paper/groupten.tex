\documentclass[modern,letterpaper]{aastex61}

% to-do list
% ----------
% - add items here
% style notes
% -----------
% - This file generates by Makefile; don't be typing ``pdflatex'' or some bullshit.
% - Line break between sentences to make the git diffs readable.
% - Use \, as a multiply operator.
% - Reserve () for function arguments; use [] or {} for outer shit.
% - Use \sectionname not Section, \figname not Figure, \documentname not Article or Paper or paper.

% packages
\definecolor{cbblue}{HTML}{3182bd}
\usepackage{microtype}  % ALWAYS!
\usepackage{amsmath,amssymb}
\usepackage{tikz}
\usepackage{aas_macros}
\hypersetup{backref,breaklinks,colorlinks,urlcolor=cbblue,linkcolor=cbblue,citecolor=black}
\graphicspath{{.}{figures/}{../notebooks/plots/}}

% define macros for text
\newcommand{\project}[1]{\textsl{#1}}
\newcommand{\acronym}[1]{{\small{#1}}}
\newcommand{\hipparcos}{Hipparcos}
\newcommand{\gaia}{\project{Gaia}}
\newcommand{\rave}{\project{\acronym{RAVE}}}
\newcommand{\apogee}{\project{\acronym{APOGEE}}}
\newcommand{\tmass}{\project{\acronym{2MASS}}}
\newcommand{\documentname}{Article}
\newcommand{\sectionname}{Section}
\newcommand{\figname}{Figure}
\newcommand{\eqname}{Equation}
\newcommand{\dr}{\acronym{DR1}}
\newcommand{\tgas}{\acronym{TGAS}}
\newcommand{\etal}{\textit{et al}.}

\newcommand{\objname}{moving group}
\newcommand{\groupTen}{Group 10}
\newcommand{\groupSixteen}{Group 16}

% define macros for math
\newcommand{\given}{\,|\,}
\newcommand{\normal}{{\mathcal{N}}}
\newcommand{\uniform}{{\mathcal{U}}}
\newcommand{\dd}{\mathrm{d}}
\newcommand{\transp}[1]{{#1}^{\!\mathsf{T}}}
\newcommand{\inv}[1]{{#1}^{-1}}
\newcommand{\bs}[1]{\boldsymbol{#1}}
\newcommand{\vperp}{\bs{v}^\perp}
\newcommand{\propm}{\bs{\mu}}
\newcommand{\mat}[1]{\mathbf{#1}}
\renewcommand{\vec}[1]{\bs{#1}}
\newcommand{\kms}{\ensuremath{\rm km~s^{-1}}}
\newcommand{\msun}{{\rm M}_\odot}
\newcommand{\data}{\mathrm{data}}
\newcommand{\snr}{[S/N]_\varpi}
\newcommand{\eye}{\mathbb{I}}
\newcommand{\absdvtan}{\ensuremath{|\Delta\vec v_\mathrm{t}|}}
\newcommand{\estimates}{\ensuremath{\{\hat{\varpi_i},\hat{\mu_{\alpha,i}},\hat{\mu_{\delta,i}},\hat{v_{r,i}}\}}}

\newcommand{\ra}{\text{R.A.}}
\newcommand{\dec}{\text{Decl.}}
\newcommand{\pmra}{\ensuremath{\mu_\alpha^*}}
\newcommand{\pmdec}{\ensuremath{\mu_\delta}}
\newcommand{\masyr}{\ensuremath{\mathrm{mas}\,\mathrm{yr}^{-1}}}

\definecolor{crimson}{rgb}{0.79, 0.0, 0.09}
\newcommand{\todo}[1]{{\color{crimson}#1}}

\newcommand{\groupDistanceEstimate}{\ensuremath{100~\mathrm{pc}}}
\newcommand{\groupAgeEstimate}{\ensuremath{500~\mathrm{Myr}}}
\newcommand{\totalNumberOfCandidates}{\ensuremath{29}}
\newcommand{\meanVelocityOpt}{\ensuremath{(u_x,\,u_y,\,u_z) = (-3.9,\,7.0,\,-3.8)~\kms}}
\newcommand{\sigVelocityOpt}{\ensuremath{0.42~\kms}}
\newcommand{\nstarsInRegion}{3413}

\begin{document}\sloppy\sloppypar\raggedbottom\frenchspacing % trust me

\title{
  A nearby young, coeval moving group
}

\author[0000-0001-7790-5308]{Semyeong Oh \etal}
\affiliation{Department of Astrophysical Sciences,
             Princeton University, Princeton, NJ 08544, USA}

\correspondingauthor{Semyeong Oh}
\email{semyeong@astro.princeton.edu}

\begin{abstract}

  We confirm and characterize a nearby (\todo{$d \approx \groupDistanceEstimate$}),
  young (\todo{age $\approx \groupAgeEstimate$}) moving group discovered
  using \gaia\ DR 1 astrometry using the updated and extended data from \gaia\ DR 2
  in combination with other photometric and spectroscopic data from the literature.

  This group, which notably includes \todo{81 Uma},
  partially overlaps with the putative new open cluster composed of
  seven A-type stars by \citet{1977ATsir.969....7L} but its status
  was dubious due to ...
  Here, we show that three of these stars are part of a larger comoving group
  that have \totalNumberOfCandidates\ candidate members.
  We model the proper motions of the candidate members with a simple
  isotropic Gaussian velocity distribution, and find that
  the velocity dispersion is very small (\todo{X}) verifying
  that they are comoving.
  Because the group spans a large area on sky, the astrometric radial velocities
  of individual stars based on the model is determined with \todo{this precision}.
  We discuss ....

\end{abstract}

\section{Introduction} % (fold)
\label{sec:introduction}

Comoving, coeval stars in the Solar neighborhood are valuable for many different
fields of astrophysics.
For stellar physics, they are important calibrator objects.


We are entreing a new era in charting the kinematic substructure and stellar
populations in the Solar neighborhood with the recent second data release (DR~2)
of \gaia\ \citep{2018arXiv180409365G}, which provides parallaxes, proper motions
and precise two-band photometry of stars brighter than $G=20$~mag as well as
radial velocities for stars brighter than 12~mag.


Previously, we performed a search for comoving pairs using the Tycho-Gaia
Astrometric Solution (\tgas), a subset of \gaia\ DR~1 which provides astrometric
measurements for 2~million source in Tycho-2 catalog ($G<12$~mag).
By looking for comoving pairs blindly, we have naturally recovered
many of the other larger known comoving stellar groups, ranging from
loose associations to open clusters.
However, we have also found some new large and small groups that
were not previously idendified.
% TODO: discuss Faherty 2018

Here, we confirm the existence of this \objname\ with \gaia\ DR 2 and
present an extended analysis of the properties.




% Stars form in group, and remain coherent in their kinematics for a period of
% time that, most importantly, depends on the mass of the initial mass of the
% molecular clouds. Thus, looking for comoving groups of stars is an efficient
% way to find moving groups of presumably single stellar population.





% \begin{figure}[ht]
%   \centering
%   \includegraphics[width=0.95\linewidth]{figures/g10_3panels.pdf}
%   \caption{\label{fig:g10_3panels}
%     Distribution of 29 stars in \groupTen\ on sky (a),
%     proper-motion space (b), and color-magnitude space (c).
%     In panel (a), the lines connecting pairs of stars show the comoving pairs
%     that constitute this group (connected component) from
%     \citet{2017AJ....153..257O}.
%     In panel (b), the ellipses show the $2\sigma$ covariances of the \tgas\
%     data.
%     In panel (c), MIST isochrones of solar-metallicity for ages of 100~Myr,
%     500~Myr, and 1~Gyr are plotted.
%   }
% \end{figure}


\subsection{Previous identifications}
\label{subsec:history}

Moving group identifications are often scattered throughout the literature. We
reviewed a (non-exhaustive) bibliography associated with all \tgas\ sources in
\groupTen\ from our previous selection \citep{2017AJ....153..257O} using the
Simbad database, and note previous identifications of this group.

Some A-type stars that were selected in \citet{2017AJ....153..257O} as potential
members of \groupTen,
HIP 66198 (81 Uma), HIP 67231 (84 Uma), and HIP 67005, are part of an open cluster
suggested by \citet{1977ATsir.969....7L} and dubbed Latyshev 2 by Archinal \& Hynes.
The reality of the putative cluster, composed of 7 A-type stars
(HIP 66198,
HIP 67231,
HIP 67005,
HIP 67848,
HIP 66738,
TYC 3851-606-1,
TYC 3469-497-1)
was dismissed in a recent review by \citet{2016IAUS..314...21M}.
However, here we report that there indeed is a comoving group containing
some of the stars in Latyshev 2, and that we have identified the lower-MS stars
in \tgas, demonstrating its physical reality more convincingly.
We found that two of the \hipparcos\ stars in Latyshev 2,
HIP 67848 (86 Uma) and HIP 66738 (83 Uma),
were missing in the \tgas\ catalog which was the parent sample of our previous search while
their proper motions have been determined by \hipparcos.
Motivated by this, we looked for any bright members of both groups that we
may have missed using \hipparcos\ stars that are not present in \tgas\ (\todo{see section X}).

% section introduction (end)

\section{Data}
\label{sec:data}

\begin{figure}
  \includegraphics[width=0.95\linewidth]{g10_sky_pm.pdf}
  \caption{Distribution of stars in the local neighborhood of \groupTen\ on sky and in proper-motion space.}
  \label{fig:distributions}
\end{figure}

We use the astrometric, photometric, and spectroscopic parameters of
\gaia\ DR~2.
Based on the angular and parallactic spread of this group from the previous
discovery using \tgas, we query the same region in \gaia\ DR 2 with generous
cuts applied in \ra, \dec, and parallax.
We go out to 20 degrees and 9 degrees from the mean position in \ra\ and \dec\
respectively, and a nominal 20~pc in distance around the median parallax.
There are \nstarsInRegion\ sources within these cuts.
Figure~\ref{fig:distributions} shows the distribution of these sources on sky
and in proper-motion space, where the black circles are
the original candidate members from \tgas.
While there is no discernable concentration of sources on sky,
a clear overdensity is found in proper-motion space at
$(\pmra,\,\pmdec)\approx(-16.8,\,-3)~\masyr$.
Of course, the sources in this overdensity do not have exactly the same proper motion
due to gemetric projection.
The span of this overdensity mainly corresponds to the span in parallax
(\figname~\ref{fig:distributions}c).
For the same 3D velocity vector, stars that are closer (larger parallaxes) will
have larger (absolute) proper motions.

In order to select candidate members of the group,
we have simply selected all stars within the ellipse marked in the lower two
panels of \figname~\ref{fig:distributions}.
We expect this selection to be neither complete nor without contamination.
However, given the clear overdensity in proper-motion space corresponding to
this group, we expect the contamination to be low.
%TODO: some quantitative statement..
We think this selection is sufficient for a first attempt to examine the general
properties of the group.
Of 199 sources that fall in the ellipse mask in the proper-motion space,
there are 33 sources that have radial velocities (RVs) measured.
We exclude 1 source (Gaia DR2 1613400929985403520)
from this subset that has precise (standard error of 0.42~\kms) RV but is
a clear outlier, i.e., contamination.
% TODO: check coordinates and note assumed GC.
For the rest of the candidate members with RVs ($N=32$),
the median and standard deviation of the velocities in the Galactic coordinates
are $(U,\,V,\,W) = (-3.6,\,-9.1,\,-1.6)~\kms$ and
$(\sigma_U,\,\sigma_V,\,\sigma_W) = (0.59,\,0.88,\,1.24)~\kms$,
aptly described as `comoving'.
We treat these 198 sources as candidate members for the following analysis.



\begin{figure}
  \includegraphics[width=0.95\linewidth]{galactic_xyz.pdf}
  \caption{Distribution of candidate members in \groupTen\ in
    the Galactic coordinates.}
  \label{fig:galactic_xyz}
\end{figure}




% We first find the lower main-sequence stars of the Pleiades (Group 0) using the
% GPS1 data. The Pleiades is chosen as it is clearly very separated in
% proper-motion space alone (Figure 1). We apply a simple cut in proper motions
% to be within the minimum and maximum values of the TGAS-selected Group 0
% members (indicated with a blue box in Figure 1, and query 15 degrees around the
% center (median RA, Dec) of the group. There are 2095 stars satisfying the cuts
% in the GPS1 catalog. Figure 2 shows their distribution in color-magnitude
% diagrams using Gaia, 2MASS, and PanSTARRS photometry with MIST (\citealt{2016ApJ...823..102C}).

% We first find the lower main-sequence stars of the Pleiades (Group 0) using the
% GPS1 data. The Pleiades is chosen as it is clearly very separated in
% proper-motion space alone (Figure~\ref{fig:g0_pm}). We apply a simple cut in
% proper motions to be within the minimum and maximum values of the TGAS-selected
% Group 0 members (indicated with a blue box in Figure~\ref{fig:g0_pm}, and query 15
% degrees around the center (median RA, Dec) of the group. There are 2095 stars
% satisfying the cuts in the GPS1 catalog. Figure~\ref{fig:gps_gps1} shows their
% distribution in color-magnitude diagrams using Gaia, 2MASS, and PanSTARRS
% photometry with the MIST isochrone of solar metallicity and 8.5~Myrs. Although
% there is clearly contamination from background stars with this simple cuts, the
% expected over-density in the low main-sequence is well recovered (and possibly
% a few suspect WDs on the cooling sequence!).

% However, there is a noticeable disagreement between the MIST isochrone and the
% data in the red colors in that the model isochrone predict bluer colors than
% observed. This is a known problem due to incomplete spectral library for
% generating stellar spectral energy distributions from physical parameters
% (Choi+2016).


% We searched the Simbad (\citealt{2000A&AS..143....9W}) database for existing
% radial velocity measurements of the candidate members

\section{Modelling as a group}

Although the original selection was done by selecting comoving pairs, we can
also model the proper motions of stars in a group (connected component) as drawn
from a single mean velocity with a small dispersion.
\todo{rephrase:
This is essentially the moving cluster method or convergent point method,
but instead of taking procedural steps to confirm or find the existence of a
convergent point, we wish to take forward modelling approach with a probabilistic basis.}
\todo{CITE: Lindegren etc.}

The components of the model are:
\begin{itemize}
  \item We assume that the velocity $\vec{v}_i$ of the star $i$ is drawn from
    a single Gaussian component with mean (group) velocity $\vec{u}$ and
    an isotropic dispersion $\sigma_{u}$:
    $$\vec{v}_i \sim \normal(\vec{u},\,\sigma_{u}).$$
    Then, the proper motion of star $i$ is $\mat{M}_i(\alpha_i,\,\delta_i)\vec{v}_i$ where
    $\mat{M}_i$ is the rotation matrix that transforms the equatorial rectangular coordinates
    \footnote{The choice of this coordinate system is arbitrary. Many of the studies on
      moving grouops in the Solar neighborhood choose the Galactic coordinates which additionally
      depends on $(\alpha_\mathrm{GC},\,\delta_\mathrm{GC})$.
    }
    to the tangential coordinates at the location $(\mathrm{R.A.},\,\mathrm{Decl.}) = (\alpha_i,\,\delta_i)$.

  \item We assume that the noise model for the \gaia\ data is Gaussian, and
    the covariance matrix $C_i$ is given and fixed:
    $$(\tilde\varpi_i,\,\tilde\mu_{\alpha,i}^*,\,\tilde\mu_{\delta,i}) \sim
      \normal((\varpi_i,\,\mu_{\alpha,i}^*,\,\mu_{\delta,i}),\,C_i)$$

  \item \emph{Priors}: We assume a constant density prior ($p(r) \propto r^2$) and a Gaussian
    prior with 30~\kms\ dispersion for the mean velocity of the group.
    For the dispersion $\sigma_u$, we assume a uniform prior in $[0,\,50)$.
\end{itemize}


\begin{figure}
  \centering
  \includegraphics[width=0.9\linewidth]{quiver_pm.pdf}
  \caption{Proper-motion vectors (left) and residual proper motion vectors (right)
    for 29 stars in \groupTen\ on sky.
    The residuals are from subtracting the projected mean velocity (\todo{see section X}).
    The length scale in each panel marked in the upper left corner.
  }
  \label{fig:quiver_pm}
\end{figure}

\begin{figure}
  \centering
  \includegraphics[width=0.9\linewidth]{isotropic_single_mc.pdf}
  \caption{
    Posterior distribution of the mean velocity $\vec{u} = (u_x,\,u_y,\,u_z)$ and
    isotropic dispersion $\sigma_u$.
    Note priors.
  }
  \label{fig:fit}
\end{figure}

\begin{figure}
  \centering
  \includegraphics[width=0.9\linewidth]{rv_ra_id.pdf}
  \caption{
    Comparing astrometric radial velocities (blue patches) to
    measurements available from the literature (black circles).
    The stars are spread along their R.A. in the vertical direction.
    The blue patches for each star shows the posterior distribution
    of the star's radial velocity predicted from the proper motions of
    all stars in the group.
    All available measurements removing duplicated exact values from
    the literature are shown.
    In general, the the predicted astrometric RVs match one or more
    available measurements for a star.
  }
  \label{fig:rv_ra_id}
\end{figure}

We first do an optimization using L-BFGS algorithm starting from the
initial values $d_{i,0} = 1/{\tilde \varpi_0}$, a random mean velocity $\vec{u}_0$
drawn from the prior distribution, and $\sigma_u=10$~\kms.
We find that the optimized mean velocity is \meanVelocityOpt\ and the dispersion \sigVelocityOpt.
We then sample the posterior distribution of $d_i$, $\vec{u}$ and $\sigma_u$ starting
from this optimum.
\figname~\ref{fig:quiver_pm} and \figname~\ref{fig:fit} summarizes
the fit result.


Because this group spans a large area on sky,
the geometric constraint on the radial velocities of the stars is interesting.
\todo{Figure:radial velocities}
We compare this to a compilation of radial velcoity measurements from the literature.
We searched the Simbad database ...

\section{Known and probable companions to candidate members}

\section{Color-magnitude diagrams and isochrone ages}


- little or no dust reddening due to high galactic latitutde
- astrometric noise


\section{Discussion}
\label{sec:discussion}

% \begin{deluxetable}{cccccccc}
\tablecaption{Candidate members of the new moving group}
\tablehead{\colhead{Id} & \colhead{RA} & \colhead{Dec} & \colhead{$d$} & \colhead{$\mu_\alpha^*$} & \colhead{$\mu_\delta$} & \colhead{$G$} & \colhead{$G-J$}\\ \colhead{ } & \colhead{$\mathrm{{}^{\circ}}$} & \colhead{$\mathrm{{}^{\circ}}$} & \colhead{$\mathrm{pc}$} & \colhead{$\mathrm{mas\,yr^{-1}}$} & \colhead{$\mathrm{mas\,yr^{-1}}$} & \colhead{$\mathrm{mag}$} & \colhead{$\mathrm{mag}$}}
\startdata
TYC 4164-274-1 & 204.86346 & 61.06173 & 102 & -18.34204 & -4.19466 & 11.72 & 1.54 \\
TYC 3471-333-1 & 210.58975 & 52.41774 & 94 & -12.38721 & -5.52270 & 10.15 & 1.12 \\
TYC 3480-1209-1 & 223.27004 & 51.26115 & 103 & -14.19318 & -0.68498 & 9.77 & 0.97 \\
HIP 69650 & 213.82070 & 52.53591 & 96 & -17.60329 & -3.47382 & 6.57 & 0.21 \\
TYC 3489-1148-1 & 234.45997 & 51.53768 & 105 & -11.96797 & -0.04552 & 10.79 & 1.24 \\
TYC 3470-485-1 & 206.31849 & 52.24726 & 101 & -14.23060 & -1.42733 & 10.36 & 1.33 \\
TYC 3851-336-1 & 205.40212 & 53.33751 & 100 & -18.00650 & -3.34968 & 9.54 & 0.96 \\
TYC 3868-177-1 & 230.81618 & 54.84823 & 112 & -13.79247 & -1.23381 & 10.89 & 1.28 \\
TYC 3867-127-1 & 225.82014 & 59.01152 & 102 & -12.24411 & -4.24575 & 9.24 & 0.84 \\
HIP 72389 & 222.01173 & 56.15920 & 98 & -15.86543 & -1.52228 & 9.75 & 1.15 \\
TYC 3851-369-1 & 205.78236 & 54.02590 & 96 & -19.00362 & -2.86724 & 9.49 & 0.96 \\
TYC 3490-1083-1 & 237.50530 & 45.92063 & 106 & -2.20292 & -7.24028 & 10.77 & 1.17 \\
HIP 73730 & 226.07328 & 59.53505 & 112 & -13.66060 & -0.16444 & 7.38 & 0.31 \\
TYC 3471-233-1 & 211.95507 & 51.95266 & 100 & -16.80352 & -4.58756 & 11.01 & 1.34 \\
TYC 3486-1405-1 & 234.13473 & 48.27970 & 104 & -7.18765 & -2.91899 & 11.13 & 1.30 \\
TYC 3860-1483-1 & 219.85980 & 54.77406 & 91 & -17.56146 & -2.79769 & 9.62 & 1.00 \\
TYC 3877-725-1 & 240.27846 & 53.41640 & 116 & -9.55302 & 0.50717 & 10.15 & 0.98 \\
HIP 71911 & 220.63149 & 60.23096 & 106 & -16.23543 & -3.84032 & 8.02 & 0.60 \\
TYC 4174-1117-1 & 209.67123 & 63.68877 & 94 & -18.90730 & -4.20806 & 10.96 & 1.60 \\
TYC 4173-609-1 & 219.82002 & 61.93126 & 101 & -17.03522 & -3.99511 & 10.20 & 1.04 \\
HIP 74458 & 228.24118 & 56.04643 & 116 & -13.15720 & -1.18896 & 7.50 & 0.40 \\
TYC 3867-1373-1 & 222.87595 & 59.53208 & 104 & -15.36222 & -1.74196 & 11.41 & 1.41 \\
TYC 3867-281-1 & 226.10718 & 59.88078 & 107 & -13.39936 & -0.06820 & 9.46 & 1.01 \\
HIP 68637 & 210.74889 & 50.97178 & 101 & -16.44373 & -6.20977 & 6.18 & 0.11 \\
TYC 3875-762-1 & 231.92341 & 59.98704 & 112 & -13.29654 & 0.09361 & 10.79 & 1.31 \\
TYC 3850-257-1 & 201.21590 & 54.89743 & 90 & -19.01098 & -6.27073 & 7.51 & 0.51 \\
HIP 63702 & 195.81947 & 57.31521 & 98 & -17.10263 & -8.19643 & 8.94 & 0.91 \\
TYC 4180-573-1 & 226.31493 & 60.70688 & 107 & -14.94724 & 3.30343 & 11.41 & 1.73 \\
HIP 66198 & 203.53030 & 55.34841 & 95 & -19.07864 & -6.07010 & 5.65 & 0.05 \\
HIP 69275 & 212.72088 & 62.52220 & 104 & -17.16634 & -3.03405 & 8.18 & 0.70 \\
HIP 75449 & 231.20862 & 51.21025 & 104 & -9.45363 & 1.95656 & 8.71 & 0.86 \\
TYC 3851-600-1 & 207.11420 & 54.04270 & 94 & -18.28773 & -3.93385 & 10.72 & 1.29 \\
TYC 3861-1374-1 & 222.52355 & 53.63483 & 104 & -14.56375 & -1.91517 & 10.04 & 1.10 \\
TYC 3865-934-1 & 216.29630 & 57.63321 & 105 & -15.38139 & -2.48544 & 9.49 & 0.86 \\
HIP 67231 & 206.64844 & 54.43266 & 98 & -18.53300 & -4.75037 & 5.72 & 0.02 \\
HIP 69917 & 214.62966 & 52.03331 & 100 & -17.19397 & -3.12164 & 6.91 & 0.32 \\
HIP 78958 & 241.78394 & 49.08308 & 114 & -13.79672 & -2.60256 & 7.57 & 0.41 \\
TYC 3869-656-1 & 232.80915 & 53.46013 & 108 & -12.59986 & -0.73862 & 11.25 & 1.33 \\
HIP 69958 & 214.73284 & 54.86376 & 104 & -16.56509 & -2.06314 & 6.45 & 0.35 \\
HIP 77903 & 238.64981 & 49.39545 & 112 & -11.66980 & -1.48668 & 9.73 & 0.90 \\
TYC 3496-1082-1 & 237.92411 & 52.30631 & 118 & -10.69995 & -1.47975 & 11.23 & 1.33 \\
TYC 3497-1053-1 & 240.51009 & 51.33533 & 120 & -10.45759 & 0.90600 & 11.08 & 1.24 \\
TYC 3867-2-1 & 226.27980 & 57.51270 & 120 & -13.18434 & -1.65660 & 11.05 & 1.57 \\
HIP 67005 & 205.97812 & 52.06439 & 93 & -18.27010 & -5.60471 & 6.05 & 0.10 \\
HIP 69721 & 214.07256 & 58.38940 & 107 & -16.25440 & -2.90256 & 8.40 & 0.70
\enddata

\tablecomments{some shit}

\end{deluxetable}



% \begin{figure}[htpb]
%   \centering
%   \includegraphics[width=0.7\linewidth]{cmd_gjg.pdf}
%   \caption{Color-magnitude diagram of member stars.
%     For comparison, we show the MIST isochrones of a solar metallicity population \todo{cite mist}.
%     Red markers indicate infrared excess,
%     and markers wrapped in black circles indicate X-ray source (Table~\todo{tablenum}).
%   }
%   \label{fig:cmd_gjg}
% \end{figure}


\appendix

\section{Queries to the Gaia Archive}


\acknowledgements
% simbad
This research has made use of the SIMBAD database,
operated at CDS, Strasbourg, France.
% ads
This research has made use of NASA's Astrophysics Data System.
% gaia
% gaia sprints


\bibliography{refs}

\end{document}
